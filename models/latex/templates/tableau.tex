\documentclass{scrartcl}
%\documentclass{article}
\usepackage[a4paper,margin=1cm,landscape]{geometry}
\usepackage[french]{babel}
\usepackage{tabularray}
\usepackage[T1]{fontenc}
\usepackage{kpfonts}
\usepackage{fontawesome5}
\usepackage{xcolor}%gestion des couleurs dans le tableau
\definecolor{gris}{rgb}{0.95,0.95,0.95}%définition des couleurs utilisées dans le document
\definecolor{gris_fonce}{rgb}{0.57,0.57,0.57}
\definecolor{vert}{rgb}{0.03,0.5,0.1}
\definecolor{orange}{rgb}{0.93,0.4,0}
\definecolor{rouge}{rgb}{0.9,0,0}
\renewcommand{\familydefault}{\sfdefault}
\newcommand{\echec}[1]{\textcolor{rouge}{\underline{#1}}}
\newcommand{\faible}[1]{\textcolor{orange}{#1}}
\newcommand{\reussite}[1]{\textcolor{vert}{#1}}

\setlength{\parindent}{0pt}%ajout Titi pour supprimer l'alinéa

\begin{document}
\centering
\thispagestyle{empty} %pour éviter le numéro en bas de page
\begin{center} \large \VAR{titre} \end{center}
\SetTblrInner{rowsep=0pt,colsep=0pt,hlines,vlines}
%\vfil %ajout Titi pour centrer verticalement
\tiny
\begin{tblr}{colspec={Q[l,m,2.5cm] *{\VAR{nb_colonnes}}{Q[c,m,\VAR{larg_colonne} cm]}},row{even} = {gris},rows={ht=6mm,font=\tiny}}
\BLOCK{for cours in en_tete[1:]:} & \VAR{cours} \BLOCK{endfor} \\
\BLOCK{for eleve in classe}\VAR{eleve[0]}\BLOCK{for point in eleve[1:]} & \VAR{point} \BLOCK{endfor} \\
\BLOCK{endfor}

\end{tblr}
\end{document}

%#voir https://stackoverflow.com/questions/46652984/python-jinja2-latex-table